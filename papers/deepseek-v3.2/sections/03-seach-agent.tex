
\label{sec:context}
\begin{figure}[h]
    \centering
    \includegraphics[width=0.95\linewidth]{figures/search.pdf}
    \caption{
    Accuracy of Browsecomp with different test-time compute expansion strategies.
    }
    \label{fig:search}
\end{figure}

Even with extended context windows such as 128k, agentic workflows, particularly in search-based scenarios, frequently encounter maximum length limitations that prematurely truncate the reasoning process. 
This bottleneck inhibits the full realization of test-time compute potential. 
To address this, we introduce context management employing simple strategies to extend token budgets at test time, when the token usage exceeds 80\% of the context window length. 
These strategies include 
(1) \textbf{Summary}, which summarizes the overflowed trajectory and re-initiates the rollout;
(2) \textbf{Discard-75\%}, which discards the first 75\% tool call history in the trajectory to free up spaces;
(3) \textbf{Discard-all}, which resets the context by discarding all previous tool call history (similar to the new context tool~\citep{opus4.5}). 
For comparison, we also implement a parallel scaling baseline, \textbf{Parallel-fewest-step}, which samples N independent trajectories and selects the trajectory with the fewest steps.

We evaluate these strategies on the BrowseComp benchmark~\citep{wei2025browsecomp}. 
As illustrated in Figure~\ref{fig:search}, under varying compute budgets, context management leads to significant performance gains by allowing the model to scale up test-time compute, providing more space to perform additional execution steps.
For example, Summary extends the average steps to 364, achieving a performance improvement of up to 60.2. However, its overall efficiency is relatively low.
Despite its simplicity, Discard-all performs well in both efficiency and scalability, achieving a score of 67.6, comparable to parallel scaling while using significantly fewer steps.


In summary, test-time compute can be scaled either serially through context management or in parallel, both effectively extending the model's problem-solving capacity.
However, different strategies exhibit varying efficiency and scalability. 
Thus, it is crucial to account for actual compute costs when benchmarking model performance. 
Meanwhile, finding the optimal combination of serial and parallel scaling to maximize both efficiency and scalability remains a crucial direction for future work.